\lstset{ %
  backgroundcolor=\color{white},   % choose the background color; you must add \usepackage{color} or \usepackage{xcolor}
  basicstyle=\footnotesize,             % the size of the fonts that are used for the code
  breakatwhitespace=false,            % sets if automatic breaks should only happen at whitespace
  breaklines=true,                 	   % sets automatic line breaking
  captionpos=b,                    	   % sets the caption-position to bottom
  commentstyle=\color{dkgreen},   % comment style
  deletekeywords={...},            	   % if you want to delete keywords from the given language
  escapeinside={\%*}{*)},             % if you want to add LaTeX within your code
  extendedchars=true,                    % lets you use non-ASCII characters; for 8-bits encodings only, does not work with UTF-8
  frame=single,	                        % adds a frame around the code
  keepspaces=true,                         % keeps spaces in text, useful for keeping indentation of code (possibly needs columns=flexible)
  keywordstyle=\color{blue},          % keyword style
  otherkeywords={*,...},                % if you want to add more keywords to the set
  numbers=left,                               % where to put the line-numbers; possible values are (none, left, right)
  numbersep=5pt,                           % how far the line-numbers are from the code
  numberstyle=\tiny\color{gray},   % the style that is used for the line-numbers
  rulecolor=\color{black},                % if not set, the frame-color may be changed on line-breaks within not-black text (e.g. comments (green here))
  showspaces=false,                       % show spaces everywhere adding particular underscores; it overrides 'showstringspaces'
  showstringspaces=false,              % underline spaces within strings only
  showtabs=false,                           % show tabs within strings adding particular underscores
  stepnumber=1,                             % the step between two line-numbers. If it's 1, each line will be numbered
  stringstyle=\color{mauve},          % string literal style
  tabsize=2,	                                  % sets default tabsize to 2 spaces
  title=\lstname                               % show the filename of files included with \lstinputlisting; also try caption instead of title
}

\chapter{Musiksequenzen mit Hilfe von DL4J erzeugen (unfinished)}
{ ...
\section{"Idee" (unfinished)}
- LSTM Netz soll mit Beispielmusik im MIDI-Format gef"uttert werden und daraufhin Musiksequenzen erzeugen

\section{Die Eingabe (unfinished)}
\subsection{Was sind MIDI Files? (unfinished)}
- Musikdatei, Informationen als Events

\subsection{MIDI Files lesen (unfinished)}
- einlesen eines MIDI Files
- Vernachl"assigung des MIDI-Takts

\subsection{DataSets erstellen (unfinished)}
- 2 Varianten implementiert
- umwandeln der eingelesenen MIDI-Events
\subsubsection{Variante 1 (unfinished)}
...
\subsubsection{Variante 2 (unfinished)}
...
\subsubsection{Vergleich der Varianten (unfinished)}
...


\section{Das Netz (unfinished)}
%%\section{Netzwerke erstellen}


\section{Die Ausgabe (unfinished)}
- did not have the time nor computational power to experiment further with this
\subsection{MIDI Files schreiben (unfinished)}
\subsection{Ergebnisse (unfinished)}
\subsubsection{Variante 1 (unfinished)}
\subsubsection{Variante 2 (unfinished)}

\section{M"ogliche Erweiterungen und Verbesserungen (unfinished)}

} %% Ende Chapter{Musiksequenzen mit Hilfe von DL4J erzeugen}