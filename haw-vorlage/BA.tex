\documentclass[draft=false
              ,paper=a4
              ,twoside=false
              ,fontsize=11pt
              ,headsepline
              ,BCOR10mm
              ,DIV11
              ]{scrbook}
\usepackage[ngerman,english]{babel}
%% see http://www.tex.ac.uk/cgi-bin/texfaq2html?label=uselmfonts
\usepackage[ansinew]{inputenc}
%%\usepackage[latin1]{inputenc}
\usepackage{libertine}
\usepackage{pifont}
\usepackage{microtype}
\usepackage{textcomp}
\usepackage[german,refpage]{nomencl}
\usepackage{setspace}
\usepackage{makeidx}
\usepackage{listings}

%%\usepackage{natbib}
\usepackage[backend=biber]{biblatex}
%%\usepackage[backend=biber, style=numeric]{biblatex}
\addbibresource{biblio.bib}

\usepackage[ngerman,colorlinks=false]{hyperref}
\usepackage{soul}
\usepackage{hawstyle}

\usepackage{wrapfig}

%% for code snippets
\usepackage{listings}
\usepackage{color}

\definecolor{dkgreen}{rgb}{0,0.6,0}
\definecolor{gray}{rgb}{0.5,0.5,0.5}
\definecolor{mauve}{rgb}{0.58,0,0.82}
%% end code snippet

%% define some colors
\colorlet{BackgroundColor}{gray!20}
\colorlet{KeywordColor}{blue}
\colorlet{CommentColor}{black!60}
%% for tables
\colorlet{HeadColor}{gray!60}
\colorlet{Color1}{blue!10}
\colorlet{Color2}{white}

%% configure colors
\HAWifprinter{
  \colorlet{BackgroundColor}{gray!20}
  \colorlet{KeywordColor}{black}
  \colorlet{CommentColor}{gray}
  % for tables
  \colorlet{HeadColor}{gray!60}
  \colorlet{Color1}{gray!40}
  \colorlet{Color2}{white}
}{}
\lstset{%
  numbers=left,
  numberstyle=\tiny,
  stepnumber=1,
  numbersep=5pt,
  basicstyle=\ttfamily\small,
  keywordstyle=\color{KeywordColor}\bfseries,
  identifierstyle=\color{black},
  commentstyle=\color{CommentColor},
  backgroundcolor=\color{BackgroundColor},
  captionpos=b,
  fontadjust=true
}
\lstset{escapeinside={(*@}{@*)}, % used to enter latex code inside listings
        morekeywords={uint32_t, int32_t}
}
\ifpdfoutput{
  \hypersetup{bookmarksopen=false,bookmarksnumbered,linktocpage}
}{}

%% more fancy C++
\DeclareRobustCommand{\cxx}{C\raisebox{0.25ex}{{\scriptsize +\kern-0.25ex +}}}

\clubpenalty=10000
\widowpenalty=10000
\displaywidowpenalty=10000

% unknown hyphenations
\hyphenation{
}

%% recalculate text area
\typearea[current]{last}

\makeindex
\makenomenclature

\begin{document}
\selectlanguage{ngerman}

%%%%%
%% customize (see readme.pdf for supported values)
\HAWThesisProperties{Author={Marina Knabbe}
                    ,Title={Erzeugung von Musiksequenzen mit LSTM-Netzwerken}
                    ,EnglishTitle={Generation of music sequences with LSTM networks}
                    ,ThesisType={Bachelorarbeit}
                    ,ExaminationType={Bachelorpr"ufung}
                    ,DegreeProgramme={Bachelor of Science Technische Informatik}
                    ,ThesisExperts={Prof. Dr.-Ing. Andreas Meisel \and Prof. Dr. Wolfgang Fohl}
                    ,ReleaseDate={24. Februar 2017}
                  }

%% title
\frontmatter

%% output title page
\maketitle

\onehalfspacing

%% add abstract pages
%% note: this is one command on multiple lines
\HAWAbstractPage
%% German abstract
{Musiksequenzen, LSTM, Neuronale Netze, Maschinenlernen, DeepLearning4J, Java}%
{Das Ziel dieser Bachelorarbeit war es, ein LSTM-Netzwerk zur Erzeugung von Musiksequenzen anhand von MIDI-Dateien in der Java Bibliothek DeepLearning4J auf Implementierbarkeit zu pr"ufen. Hierf"ur wurden zwei Ans"atze der Datenverwaltung verfolgt und mehrere Experimente in der Netzwerkparametrierung durchgef"uhrt, welche die Erwartungen nicht erf"ullen konnten, aber die Schwierigkeiten der Benutzung von DeepLearning4J f"ur diese Aufgabe hervorheben.\\}
%% English abstract
{music sequences, LSTM, neural networks, machine learning, DeepLearning4J, Java}%
{The goal of this bachelor thesis was to test the implementability of an LSTM network for the generation of music sequences using MIDI files in the Java library DeepLearning4J. For this purpose, two approaches to data management were pursued and several experiments with the network parameterization were carried out, which could not meet the expectations but emphasize the difficulties of using DeepLearning4J for this task.}

\newpage
\singlespacing

\tableofcontents
\newpage
%% enable if these lists should be shown on their own page
\listoftables
\listoffigures
\renewcommand\lstlistlistingname{Quellcodeverzeichnis}
\lstlistoflistings
\renewcommand{\lstlistingname}{Quellcode}


\typeout{===== Section: nomenclature}
%% uncomment if a TOC entry is needed
%%\addcontentsline{toc}{chapter}{Glossar}
%%\renewcommand{\nomname}{Glossar}
%%\clearpage
%%\markboth{\nomname}{\nomname} %% see nomencl doc, page 9, section 4.1
%%\printnomenclature

\addchap{Glossar(unfinished)}
{
\begin{description}

\item[K"unstlichen Neuronalen Netz (KNN)] Hier muss etwas stehen um den Effekt sehen zu k{\"o}nnen

\item[Feedforward Netzwerk] und Text dahinter 

\item[Layer] und Text dahinter 

\item[Neuron] und Text dahinter 

\item[Deep Learning Netz] Hat ein Netz mehr als ein Hidden Layer 

\item[Backpropagation] und Text dahinter 

\item[Gradient Descent] und Text dahinter 

\item[Recurrent Neuronal Networks (RNN)] und Text dahinter 

\item[Backpropagation Through Time] und Text dahinter 

\item[Gradient] Ver"anderung aller Gewichte in Bezug auf Ver"anderung im Fehler 

\item[Long Short-Term Memory (LSTM) Netze] und Text dahinter 

\item[Restricted Boltzmann Maschinen] und Text dahinter 

\item[IntelliJ IDEA] und Text dahinter 

\item[Maven] und Text dahinter 

\item[Git] und Text dahinter 

\item[Aktivierungsfunktion] und Text dahinter 

\item[DataSet] und Text dahinter 

\item[XOR] und Text dahinter 


\item[Noch ein Stichpunkt] und Text dahinter 
\item[Noch ein Stichpunkt] und Text dahinter 
\item[Noch ein Stichpunkt] und Text dahinter 
\item[Noch ein Stichpunkt] und Text dahinter 

\end{description}

Kapitel 4
...
}

%% main
\mainmatter
\onehalfspacing
%% write to the log/stdout
\typeout{===== File: chapter 1}
\chapter{Einleitung (unfinished)}
{...
} %% Ende Chapter{Einleitung (unfinished)} %% Einleitung
\typeout{===== File: chapter 2}
\chapter{K"unstliche neuronale Netze}
{
K"unstliche Intelligenz l"asst sich mit einem K"unstlichen Neuronalen Netz (KNN) realisieren. KNNs basieren auf dem Vorbild des biologischen neuronalen Netz des Gehirns.
\cite{Breitner} schreibt dazu:
\begin{quote}{\glqq}Die biologischen Vorg"ange des menschlichen Denkens und Lernens (Aktivierung von Neuronen, chemische Ver"anderung von Synapsen usw.) werden, so gut wie m"oglich, mathematisch beschrieben und in Software oder Hardware modelliert.{\grqq}\end{quote}
KNNs bestehen also aus einem Satz von Algorithmen, welche Daten interpretieren. Diese Eingangsdaten sind numerisch und m"ussen meist durch Umwandlung der originalen Daten (z.B. Bilder, Text oder Musik) geschaffen werden. Sie sind dazu entwickelt anhand von Mustererkennung Daten eigenst"andig zu Gruppieren, vorgegebenen Klassen zu zuordnen oder den weiteren Verlauf vorherzusagen.

Das Training eines KNNs l"asst sich in zwei Kategorien unterteilen: das "uberwachte Lernen und das un"uberwachte Lernen. Beim "uberwachten lernen werden dem Netz Eingangs- und Ausgangsdaten gegeben anhand dessen das Netz den Zusammenhang erlernt und sp"ater in der Lage ist neue Daten entsprechend zu klassifizieren oder die n"achsten Ausgabe vorherzusagen. Beim un"uberwachten Lernen erh"alt das Netz lediglich Eingangsdaten und lernt diese anhand von "Ahnlichkeiten  zu gruppieren.

Es gibt verschiedene Arten von K"unstlichen Neuronalen Netzen und in den folgenden Abschnitten werden drei von ihnen erkl"art.


\section{Feedforward Netzwerke}
Ein Feedforward Netzwerk besteht aus Layern und Neuronen. Die Neuronen sind f"ur die Berechnungen der Ausgabe zust"andig, w"ahrend die Layer den Aufbau des Netzes bestimmen. Abbildung 2.1 zeigt einen m"oglichen Aufbau eines k"unstliches Neurons.
\renewcommand{\figurename}{Abb.}
\begin{figure}[htp]
\centering
\includegraphics[width=0.60\textwidth]{pictures/perceptron_node.png}
\caption[Feedforward Neuron]{m"ogliches Aussehen eines Feedforward Neurons (Quelle: \cite{DL4Jimg1})}
\end{figure}
Dieses Neuron besteht aus 1 bis x\textsubscript{m} Eing"angen (Inputs) mit Gewichten (Weights), einer Eingangsfunktion (Net input function), einer Aktivierungsfunktion (Activation function) und einem Ausgang (Outputs). Die zu verarbeiteten Daten werden an die Eing"ange gelegt, durch die zugeh"origen Gewichte verst"arkt oder abgeschw"acht und anschlie{\ss}end durch die Eingangsfunktion aufsummiert. Die entstandene Summe wird dann an die Aktivierungsfunktion "ubergeben, welche das Ergebnis dieses Neurons festlegt.

Ein Layer besteht aus einer Reihe von Neuronen beliebiger Anzahl. Ein k"unstliches neuronales Netz setze sich aus einem Input Layer, einem Output Layer und beliebig vielen Hidden Layern zusammen. Hat ein Netz mehr als ein Hidden Layer so wird es auch als Deep Learning Netz bezeichnet.
\renewcommand{\figurename}{Abb.}
\begin{figure}[htp]
\centering
\includegraphics[width=0.50\textwidth]{pictures/mlp.png}
\caption[Aufbau eines Feedforward Netzes]{Aufbau eines Feedforward Netzes (Quelle: \cite{DL4Jimg1})} 
\end{figure}
Abbildung 2.2 zeigt ein Feedforward Netzwerk. Bei diesem Netz besitzt das Input Layer drei Neuronen, das Hidden Layer hat vier und das Output Layer hat zwei Neuronen. Die Ergebnisse des Input und Hidden Layers dienen dem nachfolgenden Layer als Eingang. Die Eing"ange des Input Layers und der Ausgang des Output Layers sind hier nicht dargestellt, da der Fokus auf der inneren Verkn"upfung liegen soll. Jedes Neuron hat hier so viel Ausg"ange wie die Anzahl der Neuronen im folgenden Layer und ist somit vollst"andig verkn"upft. Dies muss nicht immer der Fall sein, doch auf diesen Sonderfall soll hier nicht weiter eingegangen werden.


\subsection{Training durch Backpropagation}
Ein neuronales Netz kann anhand von Trainingsdaten eine Funktion erlernen, indem es die Gewichte ver"andert. Um sinnvolle Ergebnisse zu erhalten m"ussen die Gewichte solange angepasst werden, bis der Fehler zwischen Netzausgabe und tats"achlichen Ausgabewert am kleinsten ist. Dies wird mit Hilfe der Backpropagation gemacht, indem R"uckw"arts vom Fehler "uber die Ausg"ange, die Gewichte und die Eing"ange der verschiedenen Layer ein Zusammenhang zwischen Fehlergr"o{\ss}e und einzelnen Gewichtseinstellungen hergestellt wird. F"ur die Bestimmung der ben"otigten Gewichten benutzt man Optimierungsfunktionen. Eine weitverbreitete Optimierungsfunktion hei{\ss}t Gradient Descent. Sie beschreibt das Verh"altnis des Fehlers zu einem einzelnen Gewicht und wie sich der Fehler ver"andert wenn das Gewicht angepasst wird.

Das Ziel ist m"oglichst schnell den Punkt zu erreichen an dem der Fehler am kleinsten ist. Um dies zu erreichen wiederholt das Netz so oft wie n"otig die folgenden Schritte: Ergebnis anhand der aktuellen Gewichte bestimmen, Fehler messen, Gewichte aktualisieren.


\section{Recurrent Netzwerke}
Recurrent Neuronal Networks (RNN) betrachten im Gegensatz zu Feedforward Netzwerken nicht nur die aktuellen Eingangsdaten sondern auch die vorhergegangenen. Sie besitzen daher zwei Eingangsgr"o{\ss}en, n"amlich die gerade angelegten und die zur"uckgeleiteten aus dem vorherigen Zeitschritt.
\renewcommand{\figurename}{Abb.}
\begin{figure}[htp]
%%\begin{floatingfigure}[r]{textwidth}
\centering
\includegraphics[width=0.60\textwidth]{pictures/RNN-unrolled.png}
\caption[RNN Neuron]{vereinfachte Darstellung eines RNN Neurons (Quelle: \cite{OlahImg})}
%%\end{floatingfigure} 
\end{figure}
Abbildung 2.3 zeigt links eine vereinfachte Darstellung eines RNN Neurons mit R"uckf"uhrung, aber ohne dargestellte Gewichte oder Aktivierungsfunktion. Rechts ist das Ganze als zeitlicher Verlauf dargestellt. Im ersten Zeitschritt wird x\textsubscript{0} an den Eingang gelegt und h\textsubscript{0} als Ergebnis berechnet. Au{\ss}erdem f"uhrt ein Pfeil zum Neuron im zweiten Zeitschritt und dient dort als zweiter Eingang. Das Ergebnis, das ein Neuron liefert ist also immer vom vorherigen abh"angig. Man bezeichnet dies auch als Ged"achtnis des Netzes. Einem Netz ein Ged"achtnis zu geben macht immer dann Sinn, wenn die Eingangsdaten eine Sequenz bilden und nicht komplett unabh"angig von einander sind. Im Gegensatz zu den Feedforward Netzen k"onnen Recurrent Netzwerke Sequenzen erfassen und sie zur Erzeugung ihrer Ausgaben nutzen. Dies ist zum Beispiel bei der automatischen Textgenerierung hilfreich, wo ein folgender Buchstabe immer vom vorherigen abh"angt und nicht willk"urlich gew"ahlt werden kann. Ein RNN ist in der Lage gezielt auf ein q ein u folgen zu lassen um sinnvolle W"orter zu bilden, ein Feedforward Netzwerk kann das nicht.

\subsection{Training durch Backpropagation Through Time}
Da bei Recurrent Netzen das Ergebnis und somit der Fehler nicht nur vom aktuellen Zeitschritt abh"angt, muss auch die Backpropagation erweitert werden um sinnvoll arbeiten zu k"onnen. Backpropagation Through Time (BPTT) erg"anzt die normale Backpropagation um den Faktor Zeit, so dass ein Einfluss auf den Fehler von einem Gewicht aus fr"uheren Schritten ermittelt werden kann. Abbildung 2.4 soll diesen Vorgang verdeutlichen. 
\renewcommand{\figurename}{Abb.}
\begin{figure}[hb]
\centering
\includegraphics[width=0.65\textwidth]{pictures/bptt_cut.png}
\caption[BPTT]{entrolltes RNN f"ur BPTT (Quelle: \cite{BPTT})}
\end{figure}
Sie zeigt ein Recurrent Netz das um drei Zeitschritte entrollt wurde indem Komponenten dupliziert wurden. Dadurch l"osen sich die R"uckf"uhrungen auf und das Netzwerk verh"alt sich wie ein Feedforward Netz. Der Einfluss jedes Gewichts kann nun anteilig berechnet und anschlie{\ss}end summiert werden, so dass ein einzelner Wert je Gewicht f"ur die Anpassung ermittelt wird.

Dieses Verfahren ben"otigt nat"urlich mehr Speicher, da alle vorherigen Zust"ande und Daten f"ur eine bestimme Anzahl an Zeitschritten gespeichert werden m"ussen.


\subsection{Problem der verschwindenden und explodierenden Gradienten}
Der Gradient stellt die Ver"anderung aller Gewichte in Bezug auf Ver"anderung im Fehler dar. Wenn der Gradient unbekannt ist, ist eine Ver"anderung an den Gewichten zur Verkleinerung des Fehlers nicht m"oglich und das Netz ist nicht in der Lage zu lernen. Zu unbekannten Gradienten kann es kommen, da Informationen die durch ein Deep Netz flie{\ss}en vielfach multipliziert werden. Multipliziert man einen Betrag regelm"a{\ss}ig mit einem Wert knapp gr"o{\ss}er 1 kann das Ergebnis unmessbar gro{\ss} werde und in diesem Fall spricht man von einem explodierenden Gradienten. Umgekehrt f"uhrt eine wiederholte Multiplikation eines Betrages mit einem Wert kleiner als 1 zu einem sehr kleinem Ergebnis. Der Wert kann so klein werden, dass er von einem Netz nicht mehr gelernt werden kann. Hier spricht man von einem verschwindenden Gradienten.

Das Problem der explodierenden Gradienten l"asst sich durch eine sinnvolle Obergrenze beheben. Bei den verschwindenden Gradienten sieht eine L"osung wesentlich schwieriger aus und dieses Thema ist noch immer Gegenstand der Forschung.


\subsection{Problem der Langzeit-Abh"angigkeiten}
Wie bereits erw"ahnt sind RNNs in der Lage Sequenzen zu erkennen und mit Abh"angigkeiten zu arbeiten, doch diese F"ahigkeit ist leider begrenzt. Besteht nur eine kleine zeitliche L"ucke zwischen den von einander abh"angigen Daten, ist ein RNN in der Lage diesen Zusammenhang zu erkennen und die richtigen Schl"usse zu ziehen. Wird der zeitliche Abstand zwischen Eingabe der Daten und dem Zeitpunkt an dem sie f"ur ein Ergebnis ben"otigt werden jedoch sehr gro{\ss} kann ein RNN diesen Zusammenhang nicht mehr herstellen. Als Beispiel gibt \cite{Olah} in seinem Artikel ein Sprach-Model an, welches das n"achste Wort abh"angig vom Vorherigen vorhersagt. Ein RNN ist in der Lage im Satz {\glqq}Die Wolken sind im Himmel.{\grqq} das letzte Wort vorauszusagen, da der Abstand von Himmel und Wolken sehr klein ist. Im Text {\glqq}Ich bin in Frankreich aufgewachsen. ... Ich spreche flie{\ss}end franz"osisch.{\grqq} kann der Abstand zum letzten Wort aber sehr gro{\ss} sein und die vorherigen W"orter lassen lediglich den Schluss zu das eine Sprache folgen muss. Denn Kontext, dass es sich sehr wahrscheinlich um franz"osisch handelt, erh"alt man nur durch den ersten Satz. Ein RNN kann sich aber keinen ganzen Text merken und somit hier den Zusammenhang von Frankreich und franz"osisch nicht lernen.

Um das Problem der Langzeit-Abh"angigkeiten zu l"osen, benutzt man Long Short-Term Memory Netze.


\section{Long Short-Term Memory Netze}
Long Short-Term Memory (LSTM) Netze sind eine besondere Art von Recurrent Netzwerken, die mit Langzeit-Abh"angigkeiten arbeiten k"onnen. Sie wurden so entworfen, dass sie speziell dieses Problem l"osen, denn Informationen "uber einen langen Zeitraum zu speichern ist ihr Standardverhalten und nicht etwas was m"uhsam erlernt werden muss. Sie bestehen aus Speicherzellen, in die Informationen geschrieben und wieder herausgelesen werden k"onnen. Mit Hilfe von sogenannten Gates, die ge"offnet oder geschlossen werden, entscheidet eine Zelle was gespeichert wird und wann ein Auslesen, Reinschreiben und L"oschen erlaubt ist. Diese Gates sind analog und durch eine Sigmoid-Funktion implementiert, so dass sich ein Bereich von 0 bis 1 ergibt. (Analog hat den Vorteil gegen"uber digital dass es differenzierbar ist und somit f"ur die Backpropagation geeignet.)
\renewcommand{\figurename}{Abb.}
\begin{figure}[htb]
\centering
\includegraphics[width=0.75\textwidth]{pictures/SRN-LSTM-chain.png}
\caption[Vergleich RNN und LSTM]{Vergleich RNN und LSTM (Quelle: \cite{OlahImg})}
\end{figure}
Genau wie die Eing"ange bei den Feedforward und Recurrent Netzen besitzen die Gates Gewichte. Diese Gewichte werden ebenfalls w"ahrend des Lernprozesses angepasst, so dass die Zelle lernt wann Daten eingelassen, ausgelesen oder gel"oscht werden m"ussen.

Abbildung 2.5 zeigt zum Vergleich oben ein simples Recurrent Netz und unten ein LSTM Netz. Beide sind "uber drei Zeitschritte dargestellt, wobei der zweite Schritt jeweils ihr Innenleben wiedergibt. W"ahrend beim RNN eine simple Struktur mit nur einer Funktion (gelber Kasten in der Abbildung) f"ur das Ergebnis verantwortlich ist, benutzt ein LSTM vier Funktionen. Wie diese Funktionen mit einander agieren und zu einem Ergebnis kommen wird im n"achsten Abschnitt Schrittweise erkl"art.


\subsection{Aufbau einer Speicherzelle}
\subsubsection{Zellzustand}
\begin{wrapfigure}{r}{0.45\textwidth}
  \vspace{-30pt}
  \begin{center}
    \includegraphics[width=0.45\textwidth]{pictures/LSTM3-C-line.png}
  \end{center}
  \vspace{-20pt}
  \caption[LSTM Zellzustand]{LSTM Zellzustand (Quelle: \cite{OlahImg})}
\vspace{-10pt}
\end{wrapfigure}
Der Zellzustand ist der eigentliche Speicherort oder das Ged"achtnis des LSTM. Abbildung 2.6 zeigt den Verlauf durch eine Speicherzelle. Links wird der Zellzustand vom vorherigen Zeitschritt "ubernommen und rechts an den n"achsten weitergegeben. In der Mitte sind zwei Operation, die den Zustand w"ahrend dieses Zeitschrittes ver"andern k"onnen. Welche Aufgabe sie haben folgt im Abschnitt Zellzustands Update.

\subsubsection{Forget Gate}
\begin{wrapfigure}{r}{0.45\textwidth}
  \vspace{-30pt}
  \begin{center}
    \includegraphics[width=0.45\textwidth]{pictures/LSTM3-focus-f_cut.png}
  \end{center}
  \vspace{-20pt}
  \caption[LSTM: Forget Gate]{Forget Gate (Quelle: \cite{OlahImg})}
\vspace{-10pt}
\end{wrapfigure}
Das Forget Gate entscheidet mit Hilfe der Sigmoid-Funktion welche Informationen gel"oscht werden. Es sieht sich den alten Ausgang h\textsubscript{t-1} und den neuen Eingang x\textsubscript{t} an und gibt f"ur jede Information im Zellzustand C\textsubscript{t-1} einen Wert zwischen 0 und 1 an. Eine 1 bedeutet behalte es und eine 0 vergiss bzw. l"osche es.

Der Grund f"ur das Vorhandensein einer Vergissfunktion in einem Baustein, der die Aufgabe hat sich Sachen zu merken, liegt darin dass es manchmal sinnvoll sein kann Dinge zu vergessen. Zum Beispiel kann mit ihrer Hilfe die Speicherzelle zur"uckgesetzt werden, wenn bekannt ist dass die folgenden Daten in keinem Zusammenhang zu den vorherigen stehen.

\subsubsection{Eingang}
\begin{wrapfigure}{r}{0.45\textwidth}
  \vspace{-30pt}
  \begin{center}
    \includegraphics[width=0.45\textwidth]{pictures/LSTM3-focus-i_cut.png}
  \end{center}
  \vspace{-20pt}
  \caption[LSTM: Eingangsgate]{Eingangsgate (Quelle: \cite{OlahImg})}
\vspace{-10pt}
\end{wrapfigure}
Die Entscheidung, welche Daten gespeichert werden sollen, besteht aus zwei Teilen. Das Eingangsgate ist ebenfalls eine Sigmoid-Funktion und liefert ein Ergebnis zwischen 0 und 1. Sie entscheidet welche Daten zum Zellzustand wie stark durchgelassen werden. Au{\ss}erdem bereitet eine tanh-Funktion die Daten so auf, dass sie im Zellzustand gespeichert werden k"onnen.

\subsubsection{Zellzustands Update}
\begin{wrapfigure}{r}{0.45\textwidth}
  \vspace{-30pt}
  \begin{center}
    \includegraphics[width=0.45\textwidth]{pictures/LSTM3-focus-C_cut.png}
  \end{center}
  \vspace{-20pt}
  \caption[LSTM: Zellzustands Update]{Zellzustands Update (Quelle: \cite{OlahImg})}
\vspace{-10pt}
\end{wrapfigure}
Nachdem das Forget Gate und das Eingangsgate entschieden haben was mit den Daten passieren soll, wird der Zellzustand aktualisiert. Daf"ur wird der alte Zellzustand C\textsubscript{t-1} mit dem Ergebnis f\textsubscript{t} des Forget Gates multipliziert und somit alles gel"oscht, das vergessen werden soll. Anschlie{\ss}end werden die vom Eingangsgate skalierten und von der tanh-Funktion vorbereiteten Daten zum Zellzustand addiert.

\subsubsection{Ausgabe}
\begin{wrapfigure}{r}{0.45\textwidth}
  \vspace{-30pt}
  \begin{center}
    \includegraphics[width=0.45\textwidth]{pictures/LSTM3-focus-o_cut.png}
  \end{center}
  \vspace{-20pt}
  \caption[LSTM: Ausgabe]{Ausgabe (Quelle: \cite{OlahImg})}
\vspace{-10pt}
\end{wrapfigure}
Die Ausgabe erfolgt mit Hilfe eines Ausgabegates, welches ebenfalls eine Sigmoid-Funktion ist. Der Zellzustand wird durch eine tanh-Funktion geleitet und anschlie{\ss}end mit dem Ergebnis der Sigmoid-Funktion multipliziert. Die tanh-Funktion wandelt die Werte in einen Bereich von -1 bis 1 um, welches der typische Wertebereich von KNN-Ausg"angen ist. Durch die Multiplikation kontrolliert das Ausgangsgate, ob und wie stark das Ergebnis ausgegeben wird.

\subsubsection{Zusammenfassung}
Eine Speicherstelle besteht aus einem Zellzustand, der als Ged"achtnis fungiert und drei Gates, die den Zellzustand besch"utzen und kontrollieren. Jedes Gate arbeitet mit einer Sigmoid-Funktion, die einen Wertebereich zwischen 0 und 1 ausgibt und damit die Itensit"at der Aktion bestimmt. Das Forget Gate ist f"ur das L"oschen zust"andig, das Eingangsgate "ubernimmt die Aktion das Neu-Merkens indem es neue Informationen in den Zellzustand speichert und das Ausgangsgate bestimmt die Informationen die ausgegeben werden.


\subsection{LSTM Varianten}
Nicht alle LSTM sind so aufgebaut wie bisher beschrieben. Es gibt viele durch kleine Ver"anderungen leicht abweichende Versionen. Da eine komplette Auflistung den Umfang dieser Arbeit sprengen w"urden, werden hier nur zwei Varianten vorgestellt um einen Eindruck zu vermitteln, welche M"oglichkeiten es gibt.

\subsubsection{Guckl"ocher}
\begin{wrapfigure}{r}{0.35\textwidth}
  \vspace{-40pt}
  \begin{center}
    \includegraphics[width=0.35\textwidth]{pictures/LSTM3-var-peepholes_cut.png}
  \end{center}
  \vspace{-20pt}
  \caption[LSTM Variante: Guckl"ocher]{Variante Guckl"ocher (Quelle: \cite{OlahImg})}
\vspace{-10pt}
\end{wrapfigure}
In dieser Variante werden den Gates eine Guckloch-Verbindung hinzugef"ugt. Dies erm"oglicht es den Gates einen Einblick in den aktuellen Zellzustand zu nehmen und die dadurch gewonnenen Informationen in ihre Entscheidung einflie{\ss}en zu lassen. Abbildung 2.11 zeigt diese Verbindungen f"ur alle drei Gates, jedoch ist dies nicht zwingend notwendig. Es besteht die M"oglichkeit auch nur einem oder zwei Gates diese Verbindung zu geben.

\subsubsection{Zusammengef"uhrte Gates}
\begin{wrapfigure}{r}{0.35\textwidth}
  \vspace{-40pt}
  \begin{center}
    \includegraphics[width=0.35\textwidth]{pictures/LSTM3-var-tied_cut.png}
  \end{center}
  \vspace{-20pt}
  \caption[LSTM Variante: Zusammengef"uhrte Gates]{Variante Zusammengef"uhrte Gates (Quelle: \cite{OlahImg})}
\vspace{-10pt}
\end{wrapfigure}
Eine andere Variante ist in Abbildung 2.12 dargestellt und schlie{\ss}t das Forget Gate und das Eingangsgate zu einem Gate zusammen. Diese Ver"anderung hat die Auswirkung, dass die Entscheidung was gel"oscht und was neu gespeichert wird nur noch gemeinsam getroffen werden kann. Somit kann nur etwas vergessen werden, wenn es durch neue Informationen ersetzt wird und im Umkehrschluss k"onnen neue Informationen nur gespeichert werden, wenn andere Informationen aus dem Zellzustand gel"oscht werden.


\section{Aktueller Forschungsstand (unfinished)}

\subsection{noch ohne Titel}

\subsubsection{Analyse von LSTM-Netz-Varianten (LSTM: A Search Space Odyssey)}
- In recent years, these networks have become the state-of-the-art models for a variety of machine learning problems.
- In this paper, we present the first large-scale analysis of eight LSTM variants on three representative tasks: speech recognition, handwriting recognition, and polyphonic music modeling.
- In total, we summarize the results of 5400 experimental runs (~15 years of CPU time), which makes our study the largest of its kind on LSTM networks.
- Our results show that none of the variants can improve upon the standard LSTM architecture significantly, and demonstrate the forget gate and the output activation function to be its most critical components.
- The focus of our study is to compare different LSTM variants, and not to achieve state-of-the-art results. Therefore, our experiments are designed to keep the setup simple and the comparisons fair. The vanilla LSTM is used as a baseline and evaluated together with eight of its variants. Each variant adds, removes or modifies the baseline in exactly one aspect, which allows to isolate their effect. Three different datasets from different domains are used to account for cross-domain variations.
Quelle: 1503.04069v1.pdf

\subsubsection{Bedienung dynamischer Systeme (Control of Dynamic Systems Using LSTM supported Neural Network)}
- operation of dynamic system is challenged by non-linearity, disturbances and multivariate interactions
	- well-suited for to handle this is MPC
- combination of LSTM and NN(Neural Network) to learn complex policies of MPC(Model Predictive Control)
- MPC is a multivariate control algorithm that uses an internal dynamic model of the process, history of past control moves to yield optimal control actions.
	- In MPC, the control actions are computed by solving an optimization objective that minimizes a cost function ( function of difference in target output and system output) while accounting for system dynamics ( using a prediction model) and satisfying output and control action constraints.
	- However, solving the optimization objective in real time is computationally demanding and often takes lot of time for complex systems. Moreover, MPC requires the estimation of hidden system states (hidden states characterize system dynamics) which can be challenging in complex non-linear dynamic system.
- We propose LSTM supported NN model (LSTMSNN). The output of LSTMSNN is a weighted combination of outputs from LSTM and NN. We use this combination because the current control action depends on past control actions, current system output and target output. The LSTM part of LSTMSNN takes past control actions as input. Because there is temporal dependency between control actions, and we want to use LSTM to capture it. The NN part of LSTMSNN takes current system output and target output as input. Because we want to train NN to make a decision on control action by using current system and target output.
	- Our trained LSTMSNN Model is computationally less expensive than MPC.
	- Moreover, our approach does not involve the burden of estimating the hidden states that characterizes system dynamics.
Quelle: 2\_lstmsnn.pdf

\subsubsection{LSTM-Netzwerk-basierte Merkmalextraktion (LSTM Networks for Mobile Human Activity Recognition)}
- In this paper, we propose a LSTM-based feature extraction approach to recognize human activities using tri-axial accelerometers data.
- The predominant approach to HAR is based on a sliding window procedure, where a fixed length analysis window is shifted along the signal sequence for frame extraction. Preprocessing then transforms raw signal data into feature vectors, which are subjected to statistical classifiers that eventually provide activity hypotheses.
- Activity recognition has a wide range of applications in mobile applications — from fitness and health tracking to context-based advertising and employee monitoring.
- The features used in most of researches on HAR are selected by hand. Designing hand-crafted features in a specific application requires domain knowledge [18], and maybe result in loss of information after extracting features.
Quelle: 014.pdf

\subsubsection{Semantically Conditioned LSTM-based Natural Language Generation for Spoken Dialogue Systems}
- This paper presents a statistical language generator based on a semantically controlled Long Short-term Memory (LSTM) structure. The LSTM generator can learn from unaligned data by jointly optimising sentence planning and surface realisation using a simple cross entropy training criterion, and language variation can be easily achieved by sampling from output candidates. With fewer heuristics, an objective evaluation in two differing test domains showed the proposed method improved performance compared to previous methods. Human judges scored the LSTM system higher on informativeness and naturalness and overall preferred it to the other systems.
- The most common and widely adopted today is the rule-based (or template-based) approach (Cheyer and Guzzoni, 2007; Mirkovic and Cavedon, 2011). Despite its robustness and adequacy, the frequent repetition of identical, rather stilted, output forms make talking to a rule-based generator rather tedious.
Quelle: 1508.01745v2.pdf

\subsection{Musikerzeugung}

\subsubsection{Modelling High-Dimensional Sequences with LSTM-RTRBM: Application to Polyphonic Music Generation}
- We propose an automatic music generation demo based on artificial neural networks, which integrates the ability of Long Short-Term Memory (LSTM) in memorizing and retrieving useful history information, together with the advantage of Restricted Boltzmann Machine (RBM) in high dimensional data modelling. Our model can generalize to different musical styles and generate polyphonic music better than previous models.
- In this context, we wish to combine the ability of RBM to represent a complicated distribution for each time step, together with a temporal model in sequence. We consider both long-term memory and short-term memory in our design of guide and learning modules, by increasing a bypassing channel from data source filtered by a recurrent LSTM layer and we show that our model increases performance generally.
Quelle: 582.pdf

\subsubsection{Polyphonic Music Modelling with LSTM-RTRBM}
- Our model integrates the ability of Long Short-Term Memory (LSTM) in memorizing and retrieving useful history information, together with the advantage of Restricted Boltzmann Machine (RBM) in high dimensional data modelling. Our approach greatly improves the performance of polyphonic music sequence modelling, achieving the state-of-the-art results on multiple datasets.
- For example, in order to complete a melody line, the beginning of the music sequence needs to be held in mind while the rest is played, a task which is carried out by the short-term memory. And the long-term memory will serve as the theme and emotion that will help maintain the global coherence of music. The existence of both the short-term and the longterm memory is vital for generating melodic and coherent music sequences.
Quelle: p991-lyu.pdf





https://colah.github.io/posts/2015-08-Understanding-LSTMs/
LSTMs were a big step in what we can accomplish with RNNs. It’s natural to wonder: is there another big step? A common opinion among researchers is: "Yes! There is a next step and it’s attention!" The idea is to let every step of an RNN pick information to look at from some larger collection of information. For example, if you are using an RNN to create a caption describing an image, it might pick a part of the image to look at for every word it outputs. In fact, Xu, et al. (2015) do exactly this – it might be a fun starting point if you want to explore attention! There’s been a number of really exciting results using attention, and it seems like a lot more are around the corner…
Attention isn’t the only exciting thread in RNN research. For example, Grid LSTMs by Kalchbrenner, et al. (2015) seem extremely promising. Work using RNNs in generative models – such as Gregor, et al. (2015), Chung, et al. (2015), or Bayer \& Osendorfer (2015) – also seems very interesting. The last few years have been an exciting time for recurrent neural networks, and the coming ones promise to only be more so!

http://deeplearning4j.org/lstm.html
In the mid-90s, a variation of recurrent net with so-called Long Short-Term Memory units, or LSTMs, was proposed by the German researchers Sepp Hochreiter and Juergen Schmidhuber as a solution to the vanishing gradient problem.
} %% Ende Chapter{RNN und LSTM} %% KNN
\typeout{===== File: chapter 3}
\lstset{ %
  backgroundcolor=\color{white},   % choose the background color; you must add \usepackage{color} or \usepackage{xcolor}
  basicstyle=\footnotesize,             % the size of the fonts that are used for the code
  breakatwhitespace=false,            % sets if automatic breaks should only happen at whitespace
  breaklines=true,                 	   % sets automatic line breaking
  captionpos=b,                    	   % sets the caption-position to bottom
  commentstyle=\color{dkgreen},   % comment style
  deletekeywords={...},            	   % if you want to delete keywords from the given language
  escapeinside={\%*}{*)},             % if you want to add LaTeX within your code
  extendedchars=true,                    % lets you use non-ASCII characters; for 8-bits encodings only, does not work with UTF-8
  frame=single,	                        % adds a frame around the code
  keepspaces=true,                         % keeps spaces in text, useful for keeping indentation of code (possibly needs columns=flexible)
  keywordstyle=\color{blue},          % keyword style
  otherkeywords={*,...},                % if you want to add more keywords to the set
  numbers=left,                               % where to put the line-numbers; possible values are (none, left, right)
  numbersep=5pt,                           % how far the line-numbers are from the code
  numberstyle=\tiny\color{gray},   % the style that is used for the line-numbers
  rulecolor=\color{black},                % if not set, the frame-color may be changed on line-breaks within not-black text (e.g. comments (green here))
  showspaces=false,                       % show spaces everywhere adding particular underscores; it overrides 'showstringspaces'
  showstringspaces=false,              % underline spaces within strings only
  showtabs=false,                           % show tabs within strings adding particular underscores
  stepnumber=1,                             % the step between two line-numbers. If it's 1, each line will be numbered
  stringstyle=\color{mauve},          % string literal style
  tabsize=2,	                                  % sets default tabsize to 2 spaces
  title=\lstname                               % show the filename of files included with \lstinputlisting; also try caption instead of title
}

\chapter{Deeplearning4J}
{
DeepLearning4J (DL4J) ist eine Open-Source Deep Learning Bibliothek f"ur die Java Virtual Machine. \cite{DL4J} stellt mehrere Beispielprogramme zur Verf"ugung, wie z.B. Datenklassifizierung mit Feedforward Netzwerk, XOR-Netzwerk mit manueller Erstellung von einem simplen DataSet oder zuf"allige Texterzeugung im Shakespeare Schreibstil.

Die Entwickler empfehlen eine Benutzung der Tool-Kombination IntelliJ IDEA, Apache Maven und Git f"ur ein komfortables Arbeiten und  falls ben"otigt, eine erleichterte Hilfestellung via Chat.

Dieses Kapitel ist unterteilt in die zwei Bereiche Netzwerke erstellen und Netzwerke trainieren. Im ersten Abschnitt werden Implementierungsm"oglichkeiten aufgezeigt und ein kurzer "Uberblick zu den zur Verf"ugung stehenden Layertypen gegeben. Der zweite Abschnitt befasst sich mit dem Trainieren von Netzwerken und dem besonderen Datenformat, in welches die Trainingsdaten gebracht werden m"ussen.

Wie ein neues DL4J Projekt in IntelliJ aufgesetz werden kann, wird im Anhang A erl"autert.


%%%% SECTION %%%%
\section{Netzwerke erstellen}
Ein Neuronales Netz wird in DL4J durch drei Komponeten erstellt. Die Komponenten sind der NeuralNetConfiguration.Builder, der ListBuilder und die MultiLayerConfiguration. Nachfolgend werden zwei Beispiele gegeben wie die Implementation aussehen kann. Die 3-Schritt-Methode zeigt die Implementation jeder Komponente einzeln und die Kurzversion fasst die drei Schritte zusammen, was Codezeilen spart, aber f"ur Neulinge vermutlich etwas schwerer verst"andlich ist.\\
An zu merken ist noch, dass es sich bei dem gezeigten Code nicht um dasselbe Netzwerk handelt, sondern zwei verschiedene Netzwerke gezeigt werden. Da es hier nicht um einen Vergleich der beiden Methoden geht, sondern nur gezeigt werden soll in welcher Form eine Implementation m"oglich ist. Beide Codeausz"uge stammen von den Netzwerk-Beispielen, welche \cite{DL4J} zum Download zur Verf"ugung stellt.

\subsection{Ein Netzwerk erstellen (3-Schritt-Methode)}

\subsubsection{Schritt 1: NeuralNetConfiguration.Builder}
Mit Hilfe des NeuralNetConfiguration.Builder kann man die Netzparameter festlegen. Der Quellcode \ref{lst:builder} enth"alt hierzu einen Auszug aus einem Beispielprogramm von \cite{DL4J}.
\lstinputlisting[language=JAVA, firstline=1, lastline=9,  captionpos=b, caption={NeuralNetConfiguration.Builder Beispiel}, label=lst:builder]
{code_snippets/rnn_auszug1.java}
Dieser Code enth"alt lediglich eine Auswahl aller einstellbaren Paramter, welche in der folgenden Tabelle \ref{tbl:beispieltabelle} zeilenweise erkl"art werden. F"ur Informationen zu weiteren verf"ugbaren Paramtern kann die DL4J Dokumentation zu Rate gezogen werden.

\begin{table} [h]
\begin{tabular}{|p{0.8cm}|p{3.7cm}|p{8.8cm}|}\hline
   \textbf{Zeile} & \textbf{Parameter} & \textbf{Beschreibung} \\ \hline
   2 & iterations( int ) & Anzahl der Optimierungsdurchl"aufe \\ \hline
   3 & learningRate( double ) & Lernrate (Defaulteinstellung: 1e-1) \\ \hline
   4 & optimizationAlgo( OptimizationAlgorithm ) & benutzter Optimierungsalgorithmus (z. B.: Conjugate Gradient, Hessian free, ...) \\ \hline
  5 & seed( long ) & Ursprungszahl f"ur Zufallszahlengenerator (wird zur Reproduzierbarkeit von Durchl"aufen benutzt) \\ \hline
  6 & biasInit( double ) & Initialisierung der Netzwerk Bias (Default: 0.0) \\ \hline
  7 & miniBatch( boolean ) & Eingabeverarbeitung als Minibatch oder komplettes Datenset. (Default: true) \\ \hline
  8 & updater( Updater ) & Methode zum aktuallisieren des Gradienten (z.B.: Updater.SGD = standard stochastic gradient descent) \\ \hline
   9 & weightInit( WeightInit ) & Initiallationsschema der Gewichte (z.B.: normalized, zero, ...) \\ \hline
 \end{tabular}
\caption{"Ubersicht einiger Netzwerkparameter}
\label{tbl:beispieltabelle} % Verweis im Text mittels \ref{tbl:beispieltabelle}
\end{table}

\subsubsection{Schritt 2: ListBuilder}
Mit Hilfe des erstellten NeuralNetConfiguration.Builders kann ein ListBuilder erstellt werden (siehe Quellcode \ref{lst:listbuilder}). Der ListBuilder ist f"ur die Netzstruktur zust"andig und verwaltet die Netzwerk-Layer. Beim Erstellen wird ihm die Anzahl aller verwendeten Layer mitgeteilt. (In diesem Beispiel wurde das Input Layer alse Hidden Layer mit gez"ahlt, wodurch bei der "Ubergabe der Layeranzahl lediglich das Output Layer hinzuaddiert werden muss.)

\lstinputlisting[language=JAVA, firstline=11, lastline=11,  captionpos=b, caption={Erstellen des ListBuilders}, label=lst:listbuilder]
{code_snippets/rnn_auszug1.java}
Quellcode \ref{lst:layer} zeigt das Erzeugen der einzelnen Layer f"ur ein RNN. RNNs nutzen in DL4J den GraveLSTM.Builder zum Erzeugen des Input und der Hidden Layer (siehe Zeile 2 bis 5). In Zeile 3 wird die Anzahl der Eingangsknoten "ubergeben, welche f"ur das Input Layer in diesem Beipiel die Anzahl aller zul"assigen Buchstaben ist und f"ur die Hidden Layer eine vorher festgelegt Konstante. Zeile 4 gibt die n"otigen Verbindungen zum folgenden Layer an, welche zwingend mit der Eingangsgr"o{\ss}e des n"achsten Layers "ubereinstimmen muss. Anschlie{\ss}end wird in Zeile 5 die Aktivierungsfunktion festgelegt, bevor in Zeile 6 des erstellte Layer dem ListBuilder "ubergeben wird.

Das Output Layer wird mit Hilfe des RnnOutputLayer.Builders erstellt (siehe Zeile 9 bis 12). Die "ubergebene LossFunction in Zeile 9 gibt die Methode an, mit der der Fehler zwischen Netzwerk-Ergebnis und tats"achlichem Ergebnis berechnet wird. Die Aktivierungsfunktion \glqq softmax\grqq{} in Zeile 10 normalisiert die Output Neuronen, so dass die Summe aller Ausgaben 1 ist. Zeile 12 gibt die Anzahl der Output Neuronen an, was in diesem Beispiel der Anzahl der zul"assigen Buchstaben entspricht.

\lstinputlisting[language=JAVA, firstline=13, lastline=25,  captionpos=b, caption={Erstellen der Netzwerk-Layer}, label=lst:layer]
{code_snippets/rnn_auszug1.java}

Anschlie{\ss}end kann der ListBuilder abgeschlossen werden (siehe Quellcode \ref{lst:listbuilder2}). Hierzu kann festgelegt werden, ob ein Vortrainieren stattfinden (Zeile 1) und/oder Backpropagation angewendet werden soll (Zeile 2).
\lstinputlisting[language=JAVA, firstline=28, lastline=30,  captionpos=b, caption={Fertigstellen des ListBuilder}, label=lst:listbuilder2]
{code_snippets/rnn_auszug1.java}

\subsubsection{Schritt 3: MultiLayerNetwork}
Wurde die Vorarbeit mit dem NeuralNetConfiguration.Builder und ListBuilder erledigt, kann wie im Quellcode \ref{lst:net} ein Netzwerk erstellt werden. Hierf"ur wird das MultiLayerNetwork verwendet, welches alle Informationen als MultiLayerConfigurations vom ListBuilder erh"allt. Nach der Initialisierung (Zeile 3) ist das Netzwerk bereit trainiert zu werden.
\lstinputlisting[language=JAVA, firstline=33, lastline=35,  captionpos=b, caption={Ein Netz erzeugen}, label=lst:net]
{code_snippets/rnn_auszug1.java}

MultiLayerNetwork wird in DL4J f"ur Netzwerke benutzt, die im Groben eine einzige Bearbeitungsrichtung haben (ausgenommen netztypische R"uckf"uhrungen) und Daten vom Eingang ohne Umwege zum Ausgang weiterreichen. F"ur komplexere Netzwerkarchitekturen stellt DL4J die Klasse ComputationGraph zur Verf"ugung, welche eine willk"urlich gerichtete azyklische Graphenverbingsstruktur zwischen den Layern erlaubt. Da dies in dieser Arbeit aber nicht zur Anwendung kommt, soll hier nicht weiter auf die Implementierung eingegangen werden.

\subsection{Ein Netzwerk erstellen (Kurzversion)}
In diesem Beispiel wird ein Feedforward Netzwerk mit zwei Layern erstellt. Bei der Implementation des Quellcodes \ref{lst:netShort} wird auf die Unterteilung der Komponenten verzichtet und alle ben"otigten Netzwerkparameter sowie die Netzstruktur werden direkt in die MultiLayerConfiguration geschrieben ohne Variablen f"ur den NeuralNetConfiguration.Builder und ListBuilder anzulegen.
\lstinputlisting[language=JAVA, firstline=1, lastline=19,  captionpos=b, caption={Netzwerk erstellen Kurzversion Beispiel}, label=lst:netShort]
{code_snippets/ffnn_auszug1.java}

\subsection{Layertypen}
DL4J stellt verschiedene Layerarten zum problemspezifischen Aufbau von Netzwerken zur Verf"ugung. So benutzt ein Beispiel von \cite{DL4J} sechs verschiedene Layertypen f"ur ein Netzwerk, welches jeden Frame eines Videos klassifiziert. 

F"ur ein simples Feedforward Netzwerk kann der DenseLayer.Builder() f"ur das Input und die Hidden Layer verwendet werden. Das Output Layer kann mittels OutputLayer.Builder() implementiert werden.

Zur Zeit stellt \cite{DL4J} kein spezielles Layer zur Implementierung von Recurrent Netzen zur Verf"ugung und arbeiten in ihren Beispielen mit dem GraveLSTM.Builder() f"ur das Input und die Hidden Layer. F"ur das Output Layer steht der RnnOutputLayer.Builder() zur Verf"ugung.

LSTM Netzwerke werden ebenfalls mit dem GraveLSTM.Builder() und dem RnnOutputLayer.Builder() implementiert.


%%%% SECTION %%%%
\section{Netzwerke trainieren}
\subsection{Ein Netzwerk trainieren}
Ein erstelltes Netzwerk l"asst sich durch die Methode fit() trainieren. Quellcode \ref{lst:train} zeigt die Implementierung f"ur das Netzwerkmodel net mit Trainingsdaten im Format DataSet. Ein trainieren eines Netzes ist in DL4J nur mit diesem Datenformat m"oglich und eine Umwandlung der Ursprungsdaten somit unumg"anglich.
\lstinputlisting[language=JAVA, firstline=61, lastline=61,  captionpos=b, caption={Ein Netz trainieren}, label=lst:train]
{code_snippets/rnn_auszug1.java}

\subsection{Daten erstellen}
Neuronale Netze in DL4J arbeiten mit dem Datenformat DataSet. Ein DataSet besteht aus Eingabedaten und Ausgabedaten vom Typ INDArray. Quellcode \ref{lst:data} zeigt eine Erstellung eines DataSets f"ur ein KNN, welches die XOR-Funktion erlernen soll.
\lstinputlisting[language=JAVA, firstline=1, lastline=19,  captionpos=b, caption={Daten erstellen}, label=lst:data]
{code_snippets/xor_auszug.java}
Zeile 1 erstellt ein INDArray f"ur die Eingabedaten mit den Gr"o{\ss}en 4 (Anzahl der Trainingsbeispiele) und 2 (Anzahl der Eingangsneuronen) und initialisiert dieses mit Nullen. Das gleiche geschieht in Zeile 2 f"ur die Ausgangsdaten, welche von \cite{DL4J} meist als labels benannt sind. Obwohl ein XOR eigentlich nur einen Ausgang braucht, werden in diesem Beispiel zwei verwendet, wobei Neuron 0 f"ur false steht und Neuron 1 f"ur true.\\
Zeile 5 und 6 bilden die Eingangsdaten der zwei Neuronen f"ur das erste Trainingsbeispiel. Beide Neuronen werden mit dem Wert Null belegt und somit ergibt sich durch XOR-Logik, dass das Ergebnis false sein muss. Dies ist in den Zeilen 8 und 9 umgesetz.\\
Zeilen 11 bis 14 zeigen die Implementation des zweiten Trainingsbeispiels. Hier erh"allt das Eingangsneuron 0 den Wert 1 und das Eingangsneuron 1 den Wert 0. Dadurch wird das XOR-Ergebnis true und Ausgangsneuron 1 (welches f"ur true steht) wird mit dem Wert 1 belegt.\\
Die Implementierung der Trainingsbeispiele 2 und 3 erfolgt der XOR-Logik folgend und anschlie{\ss}end wird mit den nun vollst"andigen Eingangsdaten (input) und Ausgangsdaten (labels) in Zeile 19 ein neues DataSet erzeugt.


} %% Ende Chapter{Deeplearning4j} %% DL4J
\lstset{ %
  backgroundcolor=\color{white},   % choose the background color; you must add \usepackage{color} or \usepackage{xcolor}
  basicstyle=\footnotesize,             % the size of the fonts that are used for the code
  breakatwhitespace=false,            % sets if automatic breaks should only happen at whitespace
  breaklines=true,                 	   % sets automatic line breaking
  captionpos=b,                    	   % sets the caption-position to bottom
  commentstyle=\color{dkgreen},   % comment style
  deletekeywords={...},            	   % if you want to delete keywords from the given language
  escapeinside={\%*}{*)},             % if you want to add LaTeX within your code
  extendedchars=true,                    % lets you use non-ASCII characters; for 8-bits encodings only, does not work with UTF-8
  frame=single,	                        % adds a frame around the code
  keepspaces=true,                         % keeps spaces in text, useful for keeping indentation of code (possibly needs columns=flexible)
  keywordstyle=\color{blue},          % keyword style
  otherkeywords={*,...},                % if you want to add more keywords to the set
  numbers=left,                               % where to put the line-numbers; possible values are (none, left, right)
  numbersep=5pt,                           % how far the line-numbers are from the code
  numberstyle=\tiny\color{gray},   % the style that is used for the line-numbers
  rulecolor=\color{black},                % if not set, the frame-color may be changed on line-breaks within not-black text (e.g. comments (green here))
  showspaces=false,                       % show spaces everywhere adding particular underscores; it overrides 'showstringspaces'
  showstringspaces=false,              % underline spaces within strings only
  showtabs=false,                           % show tabs within strings adding particular underscores
  stepnumber=1,                             % the step between two line-numbers. If it's 1, each line will be numbered
  stringstyle=\color{mauve},          % string literal style
  tabsize=2,	                                  % sets default tabsize to 2 spaces
  title=\lstname                               % show the filename of files included with \lstinputlisting; also try caption instead of title
}

\chapter{Musiksequenzen mit Hilfe von DL4J erzeugen}
{
Dieses Kapitel befasst sich mit der Umsetzung eines LSTM-Netzwerkes, welches anhand einer gegebenen Beispielmusik Musiksequenzen generiert und dessen Implementierung in DeepLearning4J (DL4J). Als Musikformat wurden MIDI-Dateien gew"ahlt und es wurden zwei Ans"atze zur Verwaltung und Benutzung der Daten verfolgt.

Aufgeteilt ist dieses Kapitel in die Bereiche Eingabe, Netzwerk, Ausgabe und m"ogliche Erweiterungen und Verbesserungen. Die Eingabe umfasst die Themen {\glqq}Was sind MIDI-Dateien?{\grqq}, wie wurden MIDI-Dateien f"ur diese Arbeit eingelesen (inklusive einer Erl"auterung der zwei Implementierungsans"atze) und anschlie{\ss}end in das DL4J DataSet-Format gebracht. Der Abschnitt Netzwerk erl"autert den Aufbau des verwendeten LSTM-Netzwerkes und geht kurz auf die Unterschiede ein, die durch die zwei Ans"atze der Datenverwaltung entstehen. Der Abschnitt Ausgabe geht auf die Bereiche der Ermittlung der Netzwerkausgabe, die Umwandlung der Ausgabe in MIDI-Dateien und die vom Netzwerk erzeugten Ergebnisse ein. Abschlie{\ss}end folgt eine Sammlung von Ideen, wie die Umsetzung erweitert oder verbessert werden k"onnte.


\section{Die Eingabe}

\subsection{Was sind MIDI-Dateien?}
Das Musical Instrument Digital Interface (MIDI) Format wird zum Speichern von Audiodateien benutzt. Im Gegensatz zu anderen Formaten enth"allt eine MIDI-Datei eine Liste von Ereignissen, die zum Beispiel von einer Soundkarte in entsprechende T"one umgewandelt werden k"onnen. 
\begin{quote}{\glqq}Dadurch sind die MIDI-Dateien sehr viel kleiner als digitale Audiodateien, und die Ereignisse und Kl"ange sind editierbar, wodurch die Musik neu arrangiert, editiert und interaktiv Komponiert werden kann.{\grqq} - - - \cite{ITwissen}\end{quote} 
MIDI-Dateien bestehen aus einer Sequenze, die einen oder mehrere Tracks beinhaltet, welchen wiederum Ereignisse zugeordnet sind. Diese Ereignisse k"onnen als Nachrichten ausgelesen werden und besitzen eine MIDI-Zeit, die in Ticks angegeben ist. Abbildung 4.1 zeigt den Aufbau so einer Nachricht, welche aus zwei Teilen, Status und Daten, besteht. Das Statusbyte beginnt immer mit einer 1, gefolgt von drei Bits, die die Art der Nachricht enthalten (in Abbildung 4.1 mit s gekennzeichnet). Die letzten vier Bits geben einen von 16 m"oglichen Kan"alen an (in Abbildung 4.1 mit n gekennzeichnett).
Die Datenbytes beginnen immer mit einer 0 und geben somit Raum f"ur 128  m"ogliche Werte.
\renewcommand{\figurename}{Abb.}
\begin{figure}[htp]
\centering
\includegraphics[width=0.60\textwidth]{pictures/MIDI-Message.png}
\caption[MIDI Nachricht]{MIDI Nachricht (Quelle: \cite{MIDIImg})}
\end{figure}

Tabelle \ref{tbl:midiMess} zeigt beispielhaft den Aufbau der Nachrichten {\glqq}Note aus{\grqq} und {\glqq}Note an{\grqq}. Das n im Statusbyte steht f"ur die Kanalnummer, welche Hexadezimal angegeben wird und somit einen Bereich von 0 bis F umfasst. Daten 1 enth"allt eine der 128 m"oglichen Noten und Daten 2 die Geschwindigkeit bzw. die Intensit"at mit der die Note gespielt oder losgelassen wird. Eine {\glqq}Note an{\grqq}-Nachricht mit der Geschwindigkeit 0 ist gleichbedeutend zu {\glqq}Note aus{\grqq}.
(\cite{MIDI})

\begin{table} [h]
\centering
%%\begin{tabular}{|p{0.8cm}|p{3.7cm}|p{8.8cm}|p{8.8cm}|}\hline
\begin{tabular}{|c|c|c|c|}\hline
   \textbf{Nachricht} & \textbf{Status} & \textbf{Daten 1} & \textbf{Daten 2}\\ \hline
   Note aus & 8n & Notennummer & Geschwindigkeit \\ \hline
   Note an & 9n & Notennummer & Geschwindigkeit \\ \hline
 \end{tabular}
\caption{MIDI Nachrichtenformat}
\label{tbl:midiMess} % Verweis im Text mittels \ref{tbl:midiMess}
\end{table}


\subsection{MIDI-Dateien lesen}
Zum Einlesen einer MIDI-Datei wurde eine Java-Klasse geschrieben, die die Tracks einer Sequenze durchl"auft und alle {\glqq}Note an{\grqq} und {\glqq}Note aus{\grqq}-Ereignisse in richtiger Reihenfolge abspeichert. Um die Komplexit"at dieses Projektes am Anfang zu verkleinern, wurde auf die Erfassung der MIDI-Zeit erstmal verzichtet und der Schwerpunkt auf die reine Ereignisabfolge gelegt. Geplant war diese Komponente zu ber"ucksichtigen, sobald das LSTM-Netzwerk in der Lage war eine harmonische Musiksequenze anhand von gelernten Notenfolgen zu erzeugen. Da dieses Ziel aber nicht zufriedenstellend erreicht werden konnte, blieb hierf"ur leider keine Zeit mehr.

F"ur das Abspeichern der Ereignisse wurden die folgenden zwei Ans"atze implementiert:

\subsubsection{Variante 1: Ereignisobjekt}
Die erste Idee der Datenverwaltung war die eines Ereignisobjektes. Da ein MIDI-Ereignis sich aus drei Hauptkomponenten, Note, Geschwindigkeit und Typ (an/aus) zusammensetzt, wurden diese als Variablen gew"ahlt und die Ber"ucksichtung des MIDI-Kanals vorerst ausgelassen. Quellcode \ref{lst:midiEvents} zeigt den Konstruktor f"ur so ein Objekt. In ein solches Objekt werden die Ereignisse einer MIDI-Datei komponentenweise zu Beginn eingelesen. Dadurch erh"allt man drei Listen in der Gr"o{\ss}e des geladenen Musikst"uckes, bzw. der Anzahl der Ereignisse dieses St"uckes.
\lstinputlisting[language=JAVA, firstline=1, lastline=5,  captionpos=b, caption={MidiEvents Konstruktor}, label=lst:midiEvents]
{code_snippets/midiEvents_auszug.java}
Diese Listen enthalten in vermutlich allen F"allen doppelte Werte, da es unwahrscheinlich ist, dass ein Musikst"uck nur aus unterschiedlichen Noten und Geschwindigkeiten besteht. Somit muss noch eine Ermittlung der tats"achlich benutzten Werte erfolgen. F"ur den Ereignistyp ist bekannt, dass er nur zwei Werte annehmen kann. Die Ermittlung der Anzahl der verschiedenen Noten und Geschwindigkeiten wurde wie im Quellcode \ref{lst:matrixValidData} gezeigt umgesetz.
\lstinputlisting[language=JAVA, firstline=20, lastline=30,  captionpos=b, caption={Ereignisobjekt: Listen von g"ultigen Daten}, label=lst:matrixValidData]
{code_snippets/app_auszug.java}
Die Liste der Noten (Codezeilen 1 bis 5) und die Liste der Geschwindigkeiten (Codezeilen 7 bis 11) wurden einmal komplett durchlaufen und mit Hilfe eines LinkedHashSet alle benutzten Werte je einmal erfasst. Anschlie{\ss}end erfolgt eine Abspeicherung der Werte in eine ArrayList. W"urde man diese Reduzierung nicht machen, m"usste man das Netzwerk so auslegen, dass es in der Lage ist den Zusammenhang von 32768 (128*128*2) m"oglichen Ereignissen zulernen.

Nun stehen die Eingangsdaten als Objekt zum Erstellen einer DL4J DataSets zur Verf"ugung und die Gr"o{\ss}e der Netzwerk-Layer kann durch die Anzahl der verschiedenen Ereignisse in der kompromierten Liste bestimmt werden.

\subsubsection{Variante 2: Ereignisschl"ussel}
Da die erste Variante der Implementierung nicht zum Erfolg f"uhrte, musste ein anderer Wege gefunden werden. Der zweite Ansatz verfolgt die Idee einem Ereignis eine eindeutige Zahl (in dieser Arbeit als Schl"ussel bezeichnet) zu zuweisen. Vorgesehen f"ur diesen Schl"ussel ist ein neunstelliger Integer. Dieser setzt sich zusammen aus einer zweistelligen Kanalnummer, einem einstelligen Ereignistyp, einer dreistelligen Notennummer und einer dreistelligen Geschwindigkeit. Tabelle \ref{tbl:eventKey} zeigt die verwendete Kodierung der Ereignisschl"ussel mit einem Beispiel. Das Beispiel beschreibt das Ereignis {\glqq}Note an{\grqq} f"ur die Note A5 (81) mit der Geschwindigkeit 64 auf Kanal 1.

\begin{table} [h]
\centering
%%\begin{tabular}{|p{0.8cm}|p{3.7cm}|p{8.8cm}|p{8.8cm}|}\hline
\begin{tabular}{|c|c|c|c|c|c|}\hline
   & \textbf{Kanalnr.} & \textbf{Ereignistyp} & \textbf{Notennr.} & \textbf{Geschwindigkeit}& \textbf{Schl"ussel}\\ \hline
  \textbf{Stellen} & kk & t & nnn & ggg & kktnnnggg\\ \hline
  \textbf{Beispiel} & 00 & 1 & 081 & 064 & 001081064\\ \hline
 \end{tabular}
\caption{Ereignisschl"ussel}
\label{tbl:eventKey}
\end{table}

Das Speichern der Daten aus einer MIDI-Datei erfolgt in eine ArrayList "uber Integer, wobei die Ereignisse beim Einlesen direkt in die Schl"usselform gebracht werden. Somit erh"allt man eine Liste aller Ereignisse und kann daraus ermitteln aus wie vielen verschiedenen Ereignissen die Musiksequenze besteht. Quellcode \ref{lst:schlusselValidData} zeigt diese Erfassung, wie schon bei der ersten Variante, mit Hilfe eines LinkedHashSet, wobei {\glqq}track{\grqq} die Liste aller Ereignisse ist.
\lstinputlisting[language=JAVA, firstline=9, lastline=13,  captionpos=b, caption={Ereignisschl"ussel: Liste von g"ultigen Daten}, label=lst:schlusselValidData]
{code_snippets/app_auszug.java}
Auch in dieser Variante stehen nun sowohl die Eingangsdaten zum Erstellen eines DL4J DataSets, sowie die Anzahl der verschiedenen Ereignisse zur Verf"ugung, um die Gr"o{\ss}en der Netzwerk-Layer zu bestimmen.


\subsection{DataSets erstellen}
DL4J Netzwerke arbeiten mit dem Datenformat DataSet, welches aus einem Input und einem Label (Output) bestehen. Da das Netzwerk daraufhin trainiert werden soll, dass es bei Eingabe eines Ereignisses das Folgeereignis mit der h"ochsten Wahrscheinlichkeit ausgibt, wird als Label jeweils das folgende Ereignis zugewiesen.

F"ur die Backpropagation Through Time wird von DL4J die Benutzung von 3-dimensionalen INDArrays f"ur Input und Label gefordert. Hierauf wird in der Beschreibung der zwei Varianten noch einmal eingegangen.

Anders als erwartet, war dieser Bereich einer der schwierigsten in der Umsetzung und hat viel Zeit in Anspruch genommen. Die Umwandlung von den Daten in ein Format mit dem das Netzwerk arbeiten kann, stellte ein gro{\ss}es Verst"andnis voraus, welches zu Beginn der Arbeit noch nicht vorhanden war. Dies f"uhrte zu mehreren Implementierungsfehlern und auch die sonst sehr ausf"uhrliche und verst"andliche Dokumentation von \cite{DL4J} konnte hier die Verst"andnisl"ucken nicht f"ullen. Erst die von \cite{DL4J} zur Verf"ugung gestellten Code-Beispiele vermittelten ein erstes Gef"uhl f"ur die richtige Benutzung, reichten aber nicht aus, um bereits jetzt festzustellen, dass die Variante 1 (Ereignisobjekt) wie geplant nicht umsetzbar ist.

\subsubsection{Variante 1: Ereignisobjekt}
Obwohl diese Implementierung zu keinem Ergebnis gekommen ist, soll hier trotzdem weiter darauf eingegangen werden, um auf die begangenen Fehler und gewonnenen Erkenntnisse hinzuweisen.

Quellcode \ref{lst:matrixData} zeigt die komplette Methode, die zum Erstellen eines DataSets geschrieben wurde. Zeile 3 und 4 legen die 3-dimensionalen INDArrays f"ur den Input und die Labels an. Als erster Parameter wird die Anzahl der Mini-Batches angegeben. Mini-Batches k"onnen in DL4J bei gro{\ss}er Datenmenge in externen Dateien sinnvoll sein. Mit ihnen kann man diese Menge in kleinere St"ucke zerlegen und somit in mehreren Schritten dem Netzwerk zuf"uhren. Das hat den Vorteil, dass nicht alle Daten auf einmal geladen werden m"ussen, sondern immer nur kleinere Teile. In dieser Arbeit wurde auf die Unterteilung in Mini-Batches verzichtet und der erste Parameter ist 1, denn alle Daten werden dem Netzwerk zusammen zugef"uhrt.

Der zweite Parameter ist der Platz f"ur den Eingang und wurde als
\begin{center}{\glqq}Anzahl der verschiedenen Noten{\grqq} * {\glqq}Anzahl der verschiedenen Geschwindigkeiten{\grqq} * 2 \end{center}
festgelegt. Dadurch erh"allt man die Anzahl aller m"oglichen Eing"ange.

Der dritte Parameter gibt die L"ange aller Trainingsdaten an und wurde durch die Gr"o{\ss}e einer der Listen aus dem Ereignisobjekt angegeben.
\lstinputlisting[language=JAVA, firstline=2, lastline=23,  captionpos=b, caption={Ereignisobjekt: Trainingsdaten}, label=lst:matrixData]
{code_snippets/LSTMNetwork_auszug.java}
Anschlie{\ss}end wurde eine For-Schleife zum Bef"ullen des Input und der Label benutzt, welche "uber alle Trainingsdaten iteriert. Von Zeile 7 bis Zeile 12 werden den Variablen ihre zu diesem Ereignis (bzw. dem Folgeereignis) zugeh"origen Werte zu gewiesen, mit der Besonderheit, dass das allerletzte Ereignis das erste Ereignis als Nachfolger erh"allt.

In Zeile 13 werden die Inputs in das INDArray eingetragen und in Zeile 14 die Labels. An dieser Stelle befindet sich der Fehler im Gedankengang und der Implementierung. Der zweite Parameter m"usste eindeutig einem Ereignis zu zuordnen sein, was aber durch die Verwendung des Ereignisobjektes nicht m"oglich ist. Durch diese Art der Implementierung werden verschiedene Ereignisse mit demselben Identifikator versehen, was zu einem falschen Datensatz f"uhrt. Schon hier h"atte auffallen m"ussen, dass das nicht funktioniert, was leider durch die Unsicherheiten in der DataSet-Erstellung nicht passierte.

\subsubsection{Variante 2: Ereignisschl"ussel}
Das Bemerken des Fehlers in der Variante 1 f"uhrte schlie{\ss}lich zur Idee des Ereignisschl"ussels. Da hierbei die Ereignisse nicht aus mehreren kombinierbaren Teilen bestehen, war eine eindeutige Identifikation m"oglich.

Quellcode \ref{lst:schlusselData} zeigt die Implementierung der Methode zum Erstellen eines DataSets f"ur Ereignisschl"ussel. In Zeile 3 und 4 werden die 3-dimensionalen INDArrays f"ur den Input und die Labels angelegt. Wie bei der Variante 1 wurde auf die Benutzung von Mini-Batches verzichtet und der erste Parameter ist 1. Der zweite Parameter, die Anzahl aller m"oglichen Eing"ange, ist durch die Gr"o{\ss}e der Liste mit den verschiedenen Ereignissen festgelegt. Der dritte Parameter ist die Anzahl aller Ereignisse der eingelesenen Musik.
\lstinputlisting[language=JAVA, firstline=26, lastline=43,  captionpos=b, caption={Ereignisschl"ussel: Trainingsdaten}, label=lst:schlusselData]
{code_snippets/LSTMNetwork_auszug.java}
Das Bef"ullen erfolgt mit Hilfe einer For-Schleife, die einmal "uber die Ereignisse der eingelesenen Musik iteriert. Auch hier wird dem letzten Ereignis das Erste als Nachfolger zugewiesen. In den Zeilen 9 und 10 erhalten das Input und Label INDArray ihre Werte. Anzumerken ist hier, dass das DataSet lediglich einen Index auf das Ereignis in der Auflistung der verschiedenen Ereignisse erh"allt und nicht der Ereignisschl"ussel "ubergeben wird.



\section{Das Netzwerk}
Das benutzte LSTM-Netzwerk besteht aus einem Eingangs-, einem Ausgangs- und einem Hidden Layer. Es wurden verschiedene Parametrierungen getestet, um ein bestm"ogliches Ergebnis zu erzielen. Die Varianten des Ereignisobjektes und der Ereignisschl"ussel unterscheiden sich in der Gr"o{\ss}e der  Eingangs- und Ausgangs-Layer, wie nachfolgend erl"autert.

\subsubsection{Variante 1: Ereignisobjekt}
F"ur das Eingangs- und Ausgangs Layer dieser Variante wurde die Gr"o{\ss}e, wie im Quellcode \ref{lst:matrixLayer} gezeigt, als 
\begin{center}{\glqq}Anzahl der verschiedenen Noten{\grqq} * {\glqq}Anzahl der verschiedenen Geschwindigkeiten{\grqq} * 2 \end{center}
festgelegt, was oftmals mehr als ben"otigt ist, da nicht alle m"oglichen Kombinationen in einer Musikdatei vorkommen m"ussen.
\lstinputlisting[language=JAVA, firstline=32, lastline=33,  captionpos=b, caption={Ereignisobjekt: Gr"o{\ss}e der Ein- und Ausgangslayer}, label=lst:matrixLayer]
{code_snippets/app_auszug.java}
Diese Festlegung bedeutete f"ur ein Musikst"uck (welches als Trainingsdaten benutzt wurde), die Layergr"o{\ss}e von 16 Noten * 4 Geschwindigkeiten * 2 = 128 Eing"ange/Ausg"ange.

\subsubsection{Variante 2: Ereignisschl"ussel}
In der zweiten Variante wurde die Gr"o{\ss}e, wie im Quellcode \ref{lst:schlusselLayer} gezeigt, als Anzahl aller m"oglichen Ereignisschl"ussel bestimmt.
\lstinputlisting[language=JAVA, firstline=15, lastline=16,  captionpos=b, caption={Ereignisschl"ussel: Gr"o{\ss}e der Ein- und Ausgangslayer}, label=lst:schlusselLayer]
{code_snippets/app_auszug.java}
F"ur dasselbe Musikst"uck aus Variante 1, werden hier 36 Eing"ange bzw. Ausg"ange ben"otigt.\\
\\Dies zeigt, dass die Wahl der Verwaltung der Daten eine Rolle in der Netzwerkgr"o{\ss}e spielt und gut durchdacht werden sollte.


\section{Die Ausgabe}
\subsection{Auswerten der Netzwerkausgabe}
Zum Erzeugen einer Ausgabe wird dem Netzwerk ein einzelnes Ereignis als Starteingabe gegeben, woraufhin das Netz ein Folgeereignis ausgibt. Diese Ausgabe wird dem Netzwerk wieder als Eingabe zugef"uhrt und dieser Vorgang wird wiederholt bis die gew"unscht Gr"o{\ss}e der zu erzeugenden Musiksequenz erreicht ist.

Die Auswertung der Netzwerkausgabe erfolgt in beiden Varianten durch die Ermittlung des Ausganges mit dem h"ochsten Wert.

\subsubsection{Variante 1: Ereignisobjekt}
 Der Quellcode \ref{lst:matrixOutput} zeigt die For-Schleife zum Auslesen der Netzwerkausgabe und dem Erstellen einer Musiksequenz. In Zeile 3 wird ein Array angelegt, dass durch Zeile 4 bis 6 mit den Ausgabewerten des Ausgangslayers gef"ullt wird. Anschlie{\ss}end wird in Zeile 7 eine Methode aufgerufen, welche den h"ochsten Wert ermittelt und den zugeh"origen Index des Ausganges zur"uckgibt. Somit wurde vom Netzwerk f"ur das Eingangsereignis ein Nachfolger bestimmt. Danach muss von dem konkreten Ausgang auf ein Ereignis zur"uckgeschlossen werden und hier wurde der Fehler dieser Implementierung erkannt. Da es bei der DataSet Erstellung zu keiner eindeutigen Zuordnung von Ereignissen zu Ausg"angen kam, war eine Aufschl"usselung (Zeile 10 bis 12) , welches Ereignis als Nachfolger gew"ahlt wurde, an dieser Stelle unm"oglich.
\lstinputlisting[language=JAVA, firstline=48, lastline=63,  captionpos=b, caption={Ereignisobjekt: Netzausgabe}, label=lst:matrixOutput]
{code_snippets/LSTMNetwork_auszug.java}
Erst an diesem Punkt wurde die richtige Verwendung von DataSets verstanden und der Umstieg in der Implementierung auf die zweite Variante mit dem Ereignisschl"ussel folgte.

\subsubsection{Variante 2: Ereignisschl"ussel}
F"ur die zweite Variante sieht die For-Schleife zum Auslesen der Netzwerkausgabe und dem Erstellen einer Musiksequenz wie in Quellcode \ref{lst:schlusselOutput} aus. Zeile 3 bis 7 "ubernimmt die gleiche Arbeit wie in Variante 1 und ermittelt den Ausgang mit dem Folgeereignis mit der h"ochsten Wahrscheinlichkeit. In Zeile 10 wird anhand des Ausgangsindex das zugeh"orige Ereignis aus der Liste alle m"oglichen Ereignisse ausgelesen und in eine Ausgabeliste eingetragen. An dieser Stelle entsteht durch das Durchlaufen der Schleife eine List von Ereignisschl"usseln, die sp"ater zur Erstellung einer MIDI-Datei benutzt werden kann.
\lstinputlisting[language=JAVA, firstline=66, lastline=81,  captionpos=b, caption={Ereignisschl"ussel: Netzausgabe}, label=lst:schlusselOutput]
{code_snippets/LSTMNetwork_auszug.java}
Anschlie{\ss}end wird in Zeile 13 ein INDArray  f"ur die n"achste Netzwerkeingabe erstellt. Dieses wird in Zeile 14 mit der letzten Ausgabe bef"ullt und in Zeile 15 dem Netzwerk "ubergeben, so dass das Folgeereignis dieses Ereignisses bestimmt werden kann. Dieser Vorgang wiederholt sich solange bis die Anzahl der gew"unschten Ereignisse f"ur die zu generierende Musiksequenz erreicht wurde.

\subsection{MIDI Files schreiben}
Nachdem des Netzwerk eine Liste von Ereignissen erzeugt hat, m"ussen diese noch in eine MIDI-Datei "ubersetzt werden. Da das Netzwerk darauf ausgelegt ist nur die Ereignisse zu lernen, m"ussen alle anderen Parameter, wie z.B. Tempo und Art des Instruments manuell festgelegt werden. Auch die Feslegung des Zeitpunktes zu welchen MIDI Tick (das hei{\ss}t in welchem Abstand zum vorherigen Ereignis) das Ereignis stattfindet wird nicht vom Netzwerk abgedeckt.

Sobald die Rahmenbedingungen festgelegt sind, wird die erzeugte Ereignisschl"usselliste durchlaufen und jedes Element in eine MIDI Nachricht umgewandelt, welche einem MIDI Track zugewiesen wird.


\subsection{Ergebnisse}
\subsubsection{Variante 1: Ereignisobjekt}
Der erste Ansatz der Datenverwaltung mit einem Ereignisobjekt lieferte aufgrund der speziellen DataSet-Anforderungen kein Ergebnis, da keine vollst"andige Implementierung m"oglich war.

\subsubsection{Variante 2: Ereignisschl"ussel}
Die zweite Variante, welche mit Ereignisschl"usseln arbeitet, erbrachte nur einen Teilerfolg. Das Netzwerk war in der Lage zu erlernen, dass auf ein {\glqq}Note an{\grqq}-Ereignis irgendwann ein {\glqq}Note aus{\grqq}-Ereignis folgen muss, um eine Note nicht endlos lange zu spielen. Auch das Erzeugen desselben Ereignisses direkt aufeinanderfolgend konnte vom Netzwerk oftmals als nicht sinnvoll erkannt werden.
\renewcommand{\figurename}{Abb.}
\begin{figure}[htp]
%%\begin{floatingfigure}[r]{textwidth}
\centering
\includegraphics[width=1\textwidth]{pictures/sampleMidi1.png}
\caption[Beispielausgabe 1]{Beispielausgabe 1}
%%\end{floatingfigure} 
\end{figure}

Als Trainingsdaten wurden drei unterschiedliche MIDI-Dateien verwendet. Die erste bestand aus 1040 Ereignisse und benutzte 36 verschiedenen Ereignisschl"usseln. Mit diesen Vorgaben wurden die Musiksequenzen in Abbildung 4.2 und 4.3 erzeugt. Auff"allig ist, dass das Netzwerk sich auf einzelne Noten fixiert und diese sehr oft mehrfach hintereinander spielt. Durch die Implementierung der ersten Variante war bekannt, dass in der Trainingsdatei 16 verschiedene Noten und 4 Geschwindigkeiten vorkamen. Diese Tatsache l"asst auf eine geringe Varianz in der Tonfolge schlie{\ss}en und wird als Grund vermutet, warum das Netz nicht in der Lage ist eine Melodie zu erzeugen.

\renewcommand{\figurename}{Abb.}
\begin{figure}[htp]
%%\begin{floatingfigure}[r]{textwidth}
\centering
\includegraphics[width=1\textwidth]{pictures/sampleMidi2.png}
\caption[Beispielausgabe 2]{Beispielausgabe 2}
%%\end{floatingfigure} 
\end{figure}

Die zweite Datei bestand aus 1254 Ereignissen und benutzte 417 verschiedene Ereignisschl"ussel. Es zeigte sich, dass je mehr Eing"ange benutzt wurden, um so kleiner musste die Anzahl der Backpropagation-Schritte sein, da es sonst zu Speicherengp"assen kam. Die Ausgabe des Netzes unterschied sich kaum von der Benutzung der ersten MIDI-Datei.

Die dritte Datei bestand aus 2620 Ereignissen und benutzte 1053 verschiedene Ereignisschl"ussel. Hierbei traten Speicherplatzprobleme auf und das System brach den Vorgang mit einer Fehlermeldung ({\glqq}Speicher konnte nicht allokiert werden{\grqq}) ab.

Mehrere Experimente mit verschiedenen Netzwerkparametern f"uhrten leider zu keinem zufriedenstellenden Ergebnis. Es ist unklar, ob DL4J f"ur diese Problemstellung komplett ungeeignet ist, der Implementierungsansatz schlecht gew"ahlt oder einfach nicht die richtige Parametrierung gefunden wurde. Auszuschlie{\ss}en ist auch nicht, dass der benutzte Rechner f"ur diese Aufgabe einfach "uber zu wenig Speicherplatz und Rechenleistung verf"ugt.


%%\section{M"ogliche Erweiterungen und Verbesserungen (unfinished)}
%%- DataSetIterator()
} %% Ende Chapter{Musiksequenzen mit Hilfe von DL4J erzeugen} %% Mein Projekt
\chapter{Fazit (unfinished)}
{ ...
} %% Ende Chapter{Fazit} %% Fazit

%% appendix if used
\appendix
%%\typeout{===== File: appendix}
\lstset{ %
  backgroundcolor=\color{white},   % choose the background color; you must add \usepackage{color} or \usepackage{xcolor}
  basicstyle=\footnotesize,             % the size of the fonts that are used for the code
  breakatwhitespace=false,            % sets if automatic breaks should only happen at whitespace
  breaklines=true,                 	   % sets automatic line breaking
  captionpos=b,                    	   % sets the caption-position to bottom
  commentstyle=\color{dkgreen},   % comment style
  deletekeywords={...},            	   % if you want to delete keywords from the given language
  escapeinside={\%*}{*)},             % if you want to add LaTeX within your code
  extendedchars=true,                    % lets you use non-ASCII characters; for 8-bits encodings only, does not work with UTF-8
  frame=single,	                        % adds a frame around the code
  keepspaces=true,                         % keeps spaces in text, useful for keeping indentation of code (possibly needs columns=flexible)
  keywordstyle=\color{blue},          % keyword style
  otherkeywords={*,...},                % if you want to add more keywords to the set
  numbers=left,                               % where to put the line-numbers; possible values are (none, left, right)
  numbersep=5pt,                           % how far the line-numbers are from the code
  numberstyle=\tiny\color{gray},   % the style that is used for the line-numbers
  rulecolor=\color{black},                % if not set, the frame-color may be changed on line-breaks within not-black text (e.g. comments (green here))
  showspaces=false,                       % show spaces everywhere adding particular underscores; it overrides 'showstringspaces'
  showstringspaces=false,              % underline spaces within strings only
  showtabs=false,                           % show tabs within strings adding particular underscores
  stepnumber=1,                             % the step between two line-numbers. If it's 1, each line will be numbered
  stringstyle=\color{mauve},          % string literal style
  tabsize=2,	                                  % sets default tabsize to 2 spaces
  title=\lstname                               % show the filename of files included with \lstinputlisting; also try caption instead of title
}

\chapter{DL4J-Projekt in IntelliJ aufsetzen}
{
Nach erfolgreicher Installation von IntelliJ und Maven kann ein DL4J-Projekt mithilfe von Maven eingerichtet werden. Hierf"ur w"ahlt man \glqq File\grqq{} $\rightarrow$ \glqq New\grqq{} $\rightarrow$ \glqq Project ...\grqq . 
\renewcommand{\figurename}{Abb.}
\begin{figure}[htp]
%%\begin{floatingfigure}[r]{textwidth}
\centering
\includegraphics[width=0.80\textwidth]{pictures/mavenProj.png}
\caption[\glqq New Project\grqq{} Fenster]{\glqq New Project\grqq{} Fenster}
%%\end{floatingfigure} 
\end{figure}
Es "offnet sich das \glqq New Project\grqq -Fenster, wie in Abbildung A.1 abgebildet und man w"ahlt auf der linken Seite \glqq Maven\grqq{} (in der Abbildung mit einer roten 1 versehen). Das K"astchen \glqq Create from archetype\grqq{} (in Abbildung gekennzeichnet mit 2) wird ausgew"ahlt und aus der Liste, der verf"ugbaren Typen der Typ \glqq maven-archetype-quickstart\grqq{} (in Abbildung gekennzeichnet mit 3) gew"ahlt. Danach auf \glqq Next\grqq{}  und die GroupId (Package Name) sowie ArtifactId (?) festlegen. Anschlie{\ss}end noch einen Projektname festlegen und \glqq Finish\grqq{} dr"ucken.

http://nd4j.org/getstarted.html

Nachdem Maven f"ur einen die Projektstruktur erstellt hat, muss man die neu erstellte POM.xml Datei noch anpassen. Die POM.xml enth"alt die Projektabh"angigkeiten, welche je Projekt variieren k"onnen. Hier kann man festlegen ob das Projekt auf der CPU oder GPU laufen soll.


The default backend for CPUs is nd4j-native. You can paste that into the $<$dependencies$>$ ... $<$/dependencies$>$ section of your POM like this:
\lstset{language=XML}
\begin{lstlisting}[language=XML,caption=applicationContext.xml]
 <dependency>
   <groupId>org.nd4j</groupId>
   <artifactId>nd4j-native</artifactId>
   <version>${nd4j.version}</version>
 </dependency>
\end{lstlisting}

ND4J’s version is a variable here. It will refer to another line higher in the POM, in the $<$properties$>$ ... 
$<$/properties$>$ section, specifying the nd4j version and appearing similar to this:
\begin{lstlisting}[language=XML,caption=applicationContext.xml]
  <nd4j.version>0.4-rc3.9</nd4j.version>
  <dl4j.version>0.4-rc3.10</dl4j.version>
\end{lstlisting}

The DL4J dependencies you add to the POM will vary with the nature of your project.

In addition to the core dependency, given below, you may also want to install deeplearning4j-cli for the command-line interface, deeplearning4j-scaleout for running parallel on Hadoop or Spark, and others as needed.
\begin{lstlisting}[language=XML,caption=applicationContext.xml]
	   <dependency>
	     <groupId>org.deeplearning4j</groupId>
	     <artifactId>deeplearning4j-core</artifactId>
	     <version>${dl4j.version}</version>
	   </dependency>
\end{lstlisting}

Weitere Informationen bez"uglich GPU Betrieb oder Benutzung andere Betriebssysteme k"onnen auf der \cite{ND4J} Seite nachgeschlagen werden.
}
\addtocontents{toc}{%
  \protect\addtokomafont{chapterentry}{Anhang }
  }
%%\chapter{Maven Projekt}
%%blubbi blub

% bibliography and other stuff
\backmatter

\typeout{===== Section: literature}
%% read the documentation for customizing the style
%%\nocite{*}
%%\printbibliography
\printbibheading[title={Quellenverzeichnis}]
\printbibliography[keyword=other,heading=subbibliography,title={Literatur}]
\printbibliography[keyword=images,heading=subbibliography,title={Bilder}]

%% index
\typeout{===== Section: index}
\printindex

\newpage
\HAWasurency

\end{document}
