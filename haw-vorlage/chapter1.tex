\chapter{Einleitung}
{
Diese Arbeit befasst sich mit der Erzeugung von Musiksequenzen mit LSTM-Netzwerken und ihrer Implementierung in der Java Bibliothek DeepLearning4J (DL4J). Anhand von MIDI-Beispieldateien soll das neuronale Netz in die Lage gebracht werden eigene Musiksequenzen zu erstellen.

Hierf"ur werden im zweiten Kapitel die Grundlagen der k"unstlichen neuronalen Netzen erl"autert. Untergliedert ist dieser Abschnitt in die Teilthemen Feedforward Netzwerke, Recurrent Netzwerke und Long Short-Term Memory Netze. Anschlie{\ss}end wird eine kleine "Ubersicht "uber den aktuellen Forschungsstand im Bereich der LSTM-Netzwerke gegeben.

Kapitel 3 befasst sich mit der Java Bibliothek DeepLearning4J und gibt einen Einblick in die Benutzung, in dem es Beispiele zur Netzwerkerstellung und Trainierung aufzeigt. Wie ein neues DeepLearning4J Projekt in der Entwicklungsumgebung IntelliJ aufgesetzt werden kann, ist im Anhang A am Ende dieser Arbeit zu finden.

Im vierten Kapitel geht es um die eigentliche Implementierung des LSTM-Netzwerkes in DL4J. Dieser Bereich ist in die Abschnitte Eingabe, Netzwerk und Ausgabe aufgeteilt. Die Eingabe umfasst die Gebiete der MIDI-Dateien als Trainingsdaten und die Erstellung von DataSets. Die Ausgabe besteht aus der Auswertung der Netzwerkausgabe, dem Erstellen von MIDI-Dateien und die mit der Netzwerkimplementierung erreichten Ergebnisse.

Als Abschluss folgt das gezogene Fazit.

} %% Ende Chapter{Einleitung (unfinished)}